\documentclass[12pt]{article}
\RequirePackage{style}

\title{Introduction to Analysis Notes}

\begin{document}
\maketitle

\begin{mybox}
Most proof details are documented in my handwritten notes.
This note serves as a well-organized summary of key concepts and theorems in real analysis,
and gives some hints on how to approach the proofs.
\end{mybox}

\section{Point Set Topology}

Let $(X,d)$ be a metric space. 

\begin{definition}[interior, exterior, boundary point, adherent point]
Let $E$ be a subset of $X$ and $x_0 \in X$.
    \begin{enumerate}
        \item $x_0$ be an interior point of $E$ if there exists $r > 0$ such that $B(x_0,r) \subseteq E$.
        \item $x_0$ be an exterior point of $E$ if there exists $r > 0$ such that $B(x_0,r) \subseteq X\setminus E$.
        \item $x_0$ be a boundary point of $E$ if for every $r > 0$, $B(x_0,r)\cap E\neq \emptyset$ and $B(x_0,r)\cap (X\setminus E)\neq \emptyset$.
        \item $x_0$ be an adherent point of $E$ if for every $r > 0$, $B(x_0,r)\cap E\neq \emptyset$.
    \end{enumerate}
\end{definition}

\begin{exercise}
    Express the negation of each of the above statements.
\end{exercise}

With the above definitions, collect all interior points of $E$ to form the interior of $E$, denoted by $int(E)$. 
Similarly, we can define $ext(E),\, \partial E,\, \overline{E}$.

\begin{theorem}
    $X = int(E) \cup ext(E) \cup \partial E$ and these three sets are pairwise disjoint.
\end{theorem}

\begin{theorem}
    Let E be a subset and $x_0\in X$. TFAE:
    \begin{enumerate}
        \item $x_0$ is an adherent point of $E$.
        \item There exists a sequence $\{x_n\}^\infty_{n=1}$ in $E$ that converges $x_0$.
    \end{enumerate}
\end{theorem}
\begin{pf}
\begin{itemize}
\item ($\Rightarrow$) Consider $B(x_0,\frac{1}{n})\cap E \neq \emptyset$.
\item ($\Leftarrow$) For every $r>0$, by definition of limits, $d(x_0,x_n)<r \,\, \forall n\geq N\in \mathbb{N} \Leftrightarrow x_n \in B(x_0,r)\cap E$.
\end{itemize}
\end{pf}


\begin{definition}[Open, Closed]
    Let $E$ be a subset of $X$.
\begin{enumerate}
        \item E is open in $X$ if for every $x \in E, \exists\,r > 0$ s.t. $B(x,r) \subseteq E$.
        \item E is closed in $X$ if for every $x \in E, \forall\,r > 0$ s.t. $B(x,r) \cap E \neq \emptyset$.
\end{enumerate}
We can also define as:
\begin{enumerate}
    \item E is open if $\partial E \cap E =\emptyset \Leftrightarrow E = int(E) $.
    \item E is closed if $\partial E \subseteq E \Leftrightarrow E = \overline{E} \Leftrightarrow X\setminus E$ is open.
\end{enumerate}
\end{definition}

\begin{bbox}
\begin{definition}[Limit point]
x is a limit point of $E$ if for every $r > 0$, $B(x,r)\cap E\setminus \{x\} \neq \emptyset$.

Sequence version definition: We say $x\in X$ is a limit point of the sequence $\{x_n\}^\infty_{n=1}$
if there exists a subsequence with strictly increasing indices s.t. $x_{n_k}\rightarrow\infty$ as $k\rightarrow\infty$.
That is, every neighborhood of $x$ contains infinitely many terms of the sequence.
\end{definition}

\begin{theorem}
    TFAE:
    \begin{enumerate}
        \item $E$ is closed in $X$.
        \item $E$ contains all its limit points. That is, if $x_n$ is a sequence in $E$ that converges to $x_0 \in X$, then $x_0 \in E$. 
    \end{enumerate}
\end{theorem}
\end{bbox}



\begin{pf}
With the concept of limit points, we can prove $E$ is open iff $X\setminus E$ is closed.
\begin{itemize}
\item ($\Rightarrow$) Suppose $E$ is open. Let $x$ be any limit point of $X\setminus E$. Our goal is to show $x \in X\setminus E$ (containing all limit points).
\item ($\Leftarrow$) Given $X\setminus E$ is closed. Suppose, by contradiction, $E$ is not open, i.e. \textcolor{red}{$x\in E, x\notin int(E), B(x,r)\cap(X\setminus E)\neq \emptyset$}.
So $x$ is a limit point of $X\setminus E$ but $x\notin X\setminus E$, contradicting the assumption that $X\setminus E$ is closed (containing all its limit points).
\end{itemize}
\end{pf}



\begin{dbox}
\begin{property}
Memorize the following properties:
    \begin{enumerate} 
    \item $int(X\setminus E)=ext(E)$
    \item $\partial E = \partial (X\setminus E)$
    \item $ext(E)\cap  E = \emptyset$
    \item $int(E) \subseteq E \subseteq \overline{E}$
    \item $\partial E \subseteq \overline{E}$
    \item $\overline{E} = int(E) \cup \partial E = X\setminus ext(E) = E\cup\partial E$
\end{enumerate}
\end{property}
\end{dbox}



\begin{theorem}
    $E$ is both open and closed (clopen)$\Leftrightarrow \partial E = \emptyset$.
\end{theorem}

Example: any subset is open and closed in $X$ w.r.t the discrete metric.

\begin{theorem}
If $A \cap \overline{B} = \emptyset$ and $\overline{A}\cap B= \emptyset$, 
then $int(A \cup B) = int A \cup int B$. 
Note that $\overline{A}\cap \overline{B}=\emptyset$  is a stronger condition.
\end{theorem}

\begin{corollary}
If $A \cap \overline{B} = \emptyset$ and $\overline{A}\cap B= \emptyset$, 
then $\partial(A \cup B) = \partial A \cup \partial B$. 
\end{corollary}

\begin{theorem}
Let $B(x_0, r)$ be an open ball and $\overline{B}(x_0,r)=\{x\in X: d(x,x_0)\leq r\}$ be a closed ball:
\begin{enumerate}
    \item Open ball is open.
    \item Closed ball is closed.
    \item $\overline{B(x_0,r)}\subseteq\overline{B}(x_0,r)$.
\end{enumerate}    
\end{theorem}

\begin{exercise}
    $\overline{B(x_0,r)}\supseteq\overline{B}(x_0,r)$ is not generally true. Find a counter-example (Hint: Discrete metric).
\end{exercise}


\begin{theorem}[Lipschitz Equivalent]
    If there exists positive constants $c,C>0$ such that for any $x, y\in X$,
    \[
    cd_1(x,y)\leq d_2(x,y)\leq Cd_1(x,y),
    \]
then $d_1, d_2$ are topologically equivalent.
\end{theorem}
\begin{pf}
Let $u$ be $d_1$-open and take any $x \in u,\ \exists\ r>0$ s.t. $B_{d_1}(x, r)\subseteq u.$ Consider $B_{d_2}(x,cr)$, we want to show $B_{d_2}(x,cr)\subseteq B_{d_1}(x, r).$

Let $z \in B_{d_2}(x,cr),$ that is $d_2(x,z)<rc$. Given $d_1(x,z)\leq \frac{1}{c}d_2(x,z)<\frac{1}{c}rc=r,$ we have $z \in  B_{d_1}(x, r).$ Thus, $B_{d_2}(x,cr)\subseteq B_{d_1}(x, r)\subseteq u,$ where $u$ is also $d_2$-open.

The converse argument can be proved by the right side of the inequality, which left as an exercise. 
\end{pf}


\begin{theorem}
    Let $(V,\norm{.})$ be a vector space endowed with a norm-induced metric.
    \begin{enumerate}
        \item Verify this is a metric space.
        \item $\overline{B(x_0,r)}=\overline{B}(x_0,r)$.
    \end{enumerate}
\end{theorem}


\subsection*{Relative Topology}

\begin{definition}[Relative Openness and Closedness]
    Let $(X,d)$ be a metric space, $Y\subseteq X$ and $E\subseteq Y$.
    \begin{enumerate}
        \item $E$ is (relative) open in $Y$ if for every $x\in E$, $\exists\, r>0$ s.t. $B_Y(x,r)\subseteq E$.
        \item $E$ is (relative) closed in $Y$ if for every $x\in E$, $\exists\, r>0$ s.t. $B_Y(x,r)\cap E \neq \emptyset$,
    \end{enumerate}
    where $B_Y(x,r) = B(x,r)\cap Y$.
\end{definition}

\begin{theorem}
     Let $(X,d)$ be a metric space, $Y\subseteq X$ and $E\subseteq Y$.
    \begin{enumerate}
        \item E is (relative) open in $Y$ iff $\exists\, V$ open in $X$ s.t. $E = V\cap Y$.
        \item E is (relative) closed in $Y$ iff $\exists\, K$ closed in $X$ s.t. $E = K\cap Y$.
    \end{enumerate}
\end{theorem}

\begin{pf}
\begin{itemize}
    \item Verify $V=\cup_{x\in E}B_X(x,r_x)\cap Y=E.$
\end{itemize}
\end{pf}

\begin{theorem}
    Let $A\subseteq S\subseteq T$ in a metric space $(X,d)$.
    Then $\overline{A}^S\subseteq \overline{A}^T$.
\end{theorem}

\begin{pf}
    \begin{enumerate}
        \item Consider $B_S(x_0,\epsilon)=B_T(x_0,\epsilon)$ is a relative open ball in $S$.
        \item $V\subseteq W$, if $V\neq\emptyset$ then $W\neq\emptyset$, equivalently, if $W=\emptyset$ then $V=\emptyset$.
    \end{enumerate}
\end{pf}


\section{Completeness}
Let $(X,d)$ be a metric space, and let $(Y,d|_{Y\times Y})$ be a subspace of $(X,d)$.

\begin{bbox}
 \begin{exercise}
    Warm-up question
    \begin{enumerate}
        \item Write down the definition of Cauchy sequence.
        \item A convergent sequence implies a Cauchy sequence.
        \item Cauchy sequence with a convergent subsequence implies a convergent sequence.
    \end{enumerate}
 \end{exercise}
\end{bbox}


\begin{definition}[Complete metric space]
$(X,d)$ is complete if every Cauchy sequence in $(X,d)$ is also a convergent sequence which converges to some point in \textbf{$X$}.
\end{definition}

\begin{theorem}[Complete Subspace and closedness]
    Let $(Y,d|_{Y\times Y})$ be a subspace of $(X,d)$.
    \begin{enumerate}
        \item If $(Y,d|_{Y\times Y})$ is complete, then $Y$ is closed in $X$. Note that completeness of $(X,d)$ is not required.
        \item If $(X,d)$ is complete and $Y$ is closed, then $Y$ is complete.
    \end{enumerate}
\end{theorem}


\begin{theorem}
    If $d_1$ and $d_2$ are Lipschitz equivalent in $X$, then $(X,d_1)$ is complete iff $(X,d_2)$ is complete.
\end{theorem}

\begin{remark}
    Note that if $d_1$ and $d_2$ are topologically equivalent in $X$, then the completeness invariance might not be true.
\end{remark}


\section{Compactness}
Let $(X,d)$ be a metric space. Let $Y\subseteq X$.

\begin{definition}
  $Y$ is compact if any open cover admits a finite subcover.
\end{definition}


\begin{definition}[Sequential Compactness]
  $Y$ is compact if every sequence in $Y$ has least one convergent subsequence whose limit lies in $Y$.
\end{definition}

\begin{theorem}
    A compact metric space is both complete and bounded.
\end{theorem}

\begin{pf}
\begin{enumerate}
    \item Completeness is relatively easier. Think.
    \item Boundedness: Suppose not.
    $\forall x_0\in X$ and $r>0$, $B(x_0,r)$ cannot contain X.
    Fix $x_0$ and consider $x_n\notin B(x_0,n)$ for any $n\in \mathbb{N}.$
    By compactness, we obtain a subsequence $\{x_{n_k}\}^\infty_{0}$, which converges to $x^{*}$. Then we have:
    \[
    n_k \leq d(x_{n_k}, x_0) \leq d(x_{n_k}, x^{*}) + d(x_0, x^{*}).
    \]
    LHS goes to $\infty$ but RHS is fixed, this forces a contradiction.

    Alternatively, consider $\cup_{n\geq 1}B(x,n)$ forms a open cover of X w/o finite subcover.
\end{enumerate}
\end{pf}



\begin{definition}[Totally Bounded]
    $(X,d)$ is totally bounded if $\forall\epsilon>0$, there exists a finite number of open balls $B(x_i,\epsilon),\, i=1,2,\cdots,n$ s.t. $X\subseteq \cup^n_{i=1}B(x_i,\epsilon).$ 
\end{definition}

\begin{theorem}
    $(X,d)$ is complete and totally bounded iff it is compact.
\end{theorem}

\begin{pf}
$(\Leftarrow)$
\begin{enumerate}
    \item Suppose $(X,d)$ is not totally bounded. $\exists\,\epsilon>0$ and for any finite set of points $\{x_i\}^n_{i=1}$, there exists $y\in X$ s.t. $d(y,x_i)\geq \epsilon$ for any $i=1,2,\cdots,n$.
    So we can construct a infinite sequence which is not a Cauchy $\rightarrow$ not a convergent sequence $\rightarrow$ no convergent subsequece, contradicts compactness.
    \item Direct proof might be more efficient. Fix $\epsilon>0$, $\cup_{x\in X}B(x,\epsilon) = X$, and compactness admits a finite subcover. 
\end{enumerate}
$(\Rightarrow)$
\begin{enumerate}
    \item Take any sequence $(x_n)^\infty_{n=1}$ in $X$.
    \item By total boundedness, for some $i=1,2,\cdots,m(1)$, $B(x^{(1)}_i, 1)$ contains infinitely many terms of $(x_n)^\infty_{n=1}$, denoted by $(x^{(1)}_{n_k})^\infty_{k=1}$.
    \item For some $i=1,2,\cdots,m(2)$, $B(x^{(2)}_i, \frac{1}{2})$ contains infinitely many terms of $(x^{(1)}_{n_k})^\infty_{k=1}$, denoted by $(x^{(2)}_{n_k})^\infty_{k=1}$.
    \item Check the nested condition, and shows that the diagnol subsequence $(x^{(k)}_{n_k})$.
\end{enumerate}
\end{pf}

This is a stronger version of the previous theorem since totally bounded implies bounded (exercise).




\begin{theorem}
    If $Y$ is compact, then $Y$ is closed and bounded.
    \label{cpt is closed and bd}
\end{theorem}

\begin{pf}
\begin{itemize}
    \item $Y$ is bounded in $Y$, $Y$ is also bounded in $X$. (Use relative topology)
    \item Goal: $\overline{Y}\subseteq Y$. Take an adherent point and construct a sequence. This sequence has a convergent subsequence. Use the uniqueness of limit. Then we show the inclusion.
\end{itemize}    
\end{pf}


\begin{theorem}[Heine-Borel Theorem]
    Given $\mathbb{R}^n$ is endowed with $l_p$ metric. Let $E\subseteq \mathbb{R}^n$.
    $E$ is compact iff $E$ is closed and bounded.
\end{theorem}

\begin{pf}
    $(\Rightarrow)$ is true any metric space. $(\Leftarrow)$ Use Bolzano-Weierstrass Theorem to verify sequential compact.
\end{pf}

\begin{pbox}
\begin{remark}
In general , the reverse statement might not hold.
Take $\mathbb{N}$ endowed with discrete metric. The sequence of natural numbers has no convergent subsequece. Or, the open cover $\cup_{n\in \mathbb{N}}B(n,\frac{1}{2})$ has no finite subcover.
\end{remark}
\end{pbox}

\begin{theorem}
    In a metric space $(X,d)$, subset $Y$ in $X$ is compact iff $Y$ is sequentially compact.
\end{theorem}


\begin{pf}
$(\Leftarrow)$ Suppose $Y$ has no finite subcover. Let $\cup_{\alpha\in A}V_\alpha$ forms an open cover of $Y$.
\begin{itemize}
\item Fix $y\in Y$, we know $y$ lies in some open set. Now define 
\begin{align*}
    r(y)&:=\sup\{r>0:B_d(y,r)\subseteq V_{\alpha} \text{ for some }\alpha\in A\}\\
    r_0 &:= \inf\{r(y):y\in Y\}.
\end{align*}

\item If $r_0=0$, there exists a sequence $y_n$ s.t. $0\leq r(y_n)<\frac{1}{n}$ (Common-Used technique regarding infimum).
      By compactness, we obtain $y_{n_k}\rightarrow y_0\in Y$.
      Construct an open ball $B(y_0, \epsilon)$, there exists $y_{n_j}\in B(y_0, \epsilon) \forall j\geq N$.
      We wish $B(x_{n_j},\epsilon/2)\subseteq B(y_0, \epsilon)\subseteq V_{\alpha}$ for some $\alpha$.
      Then we have:
      \[r(x_{n_j})\geq \epsilon/2,\] which contradicts $0\leq r(y_{n_j})<\frac{1}{n_j}.$
\item If $0<r_0<\infty$, for each $y\in Y,\, \frac{r_0}{2}<r(y)$ by construction.
      Construct a sequence $\{y_n\}^\infty_1$ s.t. $B(y_n,\epsilon/2)\subseteq V_{\alpha_n}$. (This works since we assume no finite subcover).
      Verify the sequence is not a Cauchy, so it has no convergent subsequence. Contradicts with compactness.
\item If $r_0=\infty$, the argument is similar to case 2. The only difference is replace $r_0/2$ with a constant, say 1.
\end{itemize}
\end{pf}


\begin{corollary}
    Let $K_1, K_2,...$ be a sequence of non-empty of compact sets of $X$ s.t.
    \(K_1\supseteq K_2\supseteq K_3\supseteq\cdots\). Then $\bigcap^{\infty}_{i=1} K_i\neq\emptyset.$
\end{corollary}

\begin{pf}
    \begin{itemize}
        \item Suppose $\bigcap^{\infty}_{i=1} K_i=\emptyset$, then we have $K_1=K_1\setminus\bigcap^{\infty}_{i=1} K_i=\bigcup^{\infty}_1 (K_1\setminus K_i).$
        \item $V_i \equiv K_1\setminus K_i = K_1\cup(X\setminus K_i)\rightarrow V_i$ is open in $K_1$. $\bigcup^{\infty}_1 (K_1\setminus K_i) =\bigcup^{\infty}_1 V_i$ forms a open cover of $K_1$.
        \item $K_1$ is compact so it admits a finite subcover, the intersection of finite $K_i$ family forces $K_i$ to be empty, contradiction.
    \end{itemize}
\end{pf}


\begin{exercise}
    The problems are similar to the previous corollary:
    \begin{enumerate}
    \item Let $(X,d)$ be a compact set and $K_1, K_2,...$ be a sequence of non-empty of closed sets of $X$ s.t.
    \(K_1\supseteq K_2\supseteq K_3\supseteq\cdots\). Then $\bigcap^{\infty}_{i=1} K_i\neq\emptyset.$
    \item  Let $(X,d)$ be a compact set and $K_1, K_2,...$ be a sequence of closed sets in $X$ s.t.
    \(\bigcap^{n}_{i=1} K_i\neq\emptyset\). Then $\bigcap^{\infty}_{i=1} K_i\neq\emptyset.$
    \end{enumerate}
\end{exercise}

\begin{theorem}
    Let $F_1,F_2,\cdots F_n$ be a finite collection of compact subsets of $X$, then $F_1\cap F_2\cap\cdots F_n$ remains compact.
\end{theorem}

The proof can be shown via the definition of open cover. With the above theorem, we can show that:

\begin{theorem}
    Every finite subset of $X$ is compact.
\end{theorem}

\begin{pf}
    Consider a (sequential) compact singleton $\{x_i\}$, and the finite union forms a finite subset which is compact.
\end{pf}

\begin{theorem}
    If $Y$ is compact and $H\subseteq Y$ is closed, then $H$ is compact.
    Note that if $H$ is compact, then $H$ is closed and bounded (Recall \eqref{cpt is closed and bd}).
\end{theorem}

\begin{pf}
    Construct an open cover of $H$, denoted by $\cup_i V_i$. $\cup_i V_i\cup (X\setminus H)$ forms a open cover of $Y$.
    $Y$ is compact, there exists a finite subcover covering $Y$ as well as $H$.
\end{pf}


\begin{theorem}
    Let $H_1,H_2,\cdots$ be a infinite collection of compact sets in $(X,d)$, then $\bigcap^{\infty}_{i=1}H_i$ is compact.
\end{theorem}
 
\begin{pf}
    \begin{itemize}
        \item  Every compact subset is closed.
        \item  Infinte intersection of closed sets remains closed.
        \item  $\bigcap^{\infty}_{i=1}H_i\subseteq H_1$, where $H_1$ is compact.
    \end{itemize}
\end{pf}

\begin{theorem}[Compactness as topologically invariant]
    Let $(X,d_X)$ and $(Y,d_Y)$ be metric spaces and $f:X\rightarrow Y$ be a continuous mapping.
    If $K\subseteq X$ is compact, then $f(K)$ is compact.
\end{theorem}


\begin{theorem}
    Let $(X,d_X)$ and $(Y,d_Y)$ be metric spaces and $f:X\rightarrow Y$ be a continuous bijection.
    If $X$ and $Y$ are compact, then $f^{-1}$ is continuous. (i.e. $f$ is a homeomorphism).
\end{theorem}

\begin{pf}
    \begin{itemize}
        \item Take any open set $u$ in $X$, $X\setminus u\subseteq X$ is compact.
        \item $f(X\setminus u)$ is compact by continuity. (Every compact set is $\cdots$ ?)
        \item The preimage of $f^{-1}(u)$: $(f^{-1})^{-1}(u)=f(u) = Y\setminus f(X\setminus u)$ is open.
    \end{itemize}
\end{pf}


\begin{bbox}
\begin{definition}[Continuity]
    $f$ is continous at $x_0$ if
    $\,\forall \epsilon>0\, \exists\, \delta>0$ s.t. $d_X(x,x_0)<\delta\rightarrow d_Y(f(x),f(x_0))<\epsilon \Leftrightarrow B_X(x_0,\delta)\subseteq f^{-1}(B_Y(x_0, \epsilon))$.

\end{definition}
\begin{theorem}
    Let $f:X\rightarrow Y$ be a mapping. TFAE:
\begin{enumerate}
    \item $f$ is continous at $x_0$.
    \item Whenever a sequence $\{x_n\}^\infty_{n=1}$ in $X$ s.t. $x_n\xrightarrow{d_X} x_0$, then $f(x_n)\xrightarrow{d_Y} f(x_0).$ (Sequential continuity)
    \item For any open set $u$ in $Y$, $f^{-1}(u)$ is also open in $X$.
    \item For any closed set $u$ in $Y$, $f^{-1}(u)$ is also closed in $X$.
\end{enumerate}
\end{theorem}
\end{bbox}


\begin{bbox}
\begin{definition}[Uniform Continuity]
    Let $f: X\rightarrow Y$ be a map btw $(X,d_X)$ and $(Y,d_Y)$.
    $f$ is uniformly continous if $\forall \epsilon>0\,\exists\,\delta>0$ s.t. for any $x,y\in X$ $d_X(x,y)<\delta\rightarrow d_Y(f(x),f(y))<\epsilon.$
    This means $\delta$ is independent of the location of $x\in X$.
\end{definition}
\end{bbox}

\begin{theorem}
    If $f$ is compact, then $f$ is continuous iff $f$ is uniformly continous.
\end{theorem}

\begin{pf}
    \begin{enumerate}
        \item Uniform continuity implies continuity is trivial.
        \item Given $f$ is continuous. Suppose $f$ is not uniformly continous. Consider a paritcular $\delta=\frac{1}{n}$, $\exists p^n, q^n\in X$ s.t. $d_X(p_n, q_n)<\frac{1}{n}$,
              but $d_Y(f(p_n), f(q_n))\geq\epsilon$
        \item Since $X$ is compact, $d_X(p_{n_k}, q_{n_k})\rightarrow 0$.
        \item Since $f$ is continuous, $d_Y(f(p_{n_k}),f(q_{n_k}))\rightarrow 0$, contradiction.
    \end{enumerate}
\end{pf}


\begin{theorem}[Exetreme Value Theorem]
    Let $(X,\mathcal{F})$ be a compact topological space and $f:X\rightarrow\mathbb{R}$ be a continous function. Then $f$ is bounded and attains its maximum and minimum on $X$.
\end{theorem}

\begin{pf}
    $f(X)$ is compact in $\mathbb{R}$ (standard topology). This proves boundedness.
    By axiom of completeness, $f(X)$ has a supremum and an infimum. By closedness, $\sup f(X), \inf f(X)\in f(X)$ (attainable).
\end{pf}


\section{Connectedness}

\begin{exercise}
    Let $(X, d)$ be a metric space.
    \begin{enumerate}
        \item Write down the definition of disconnectedness. Equivalently, $X$ is disconnected iff $X$ has a non-empty proper subset that is clopen.
    \end{enumerate}
\end{exercise}



\begin{theorem}
    Let $(X,d)$ be a non-empty subset in $\mathbb{R}$, TFAE:
    \begin{enumerate}
        \item $X$ is connected.
        \item $X$ is an interval, i.e. whenever $x,y\in X$ and $x<z<y$, then $z\in X$.
    \end{enumerate}
\end{theorem}

\begin{theorem}[Connectedness as topologically invariant]
    Let $(X,d_X)$ and $(Y,d_Y)$ be metric spaces and $f:X\rightarrow Y$ be a continuous mapping.
    If $A\subseteq X$ is connected, then $f(A)$ is connected.
\end{theorem}


\begin{theorem}[Intermediate Value Theorem]
    Let $f:X\rightarrow \mathbb{R}$ be continuous. Let $E\subseteq X$ be a connected subset and $a,b\in E$. If $f(a)<c<f(b)$, then $\exists\, x\in (a,b)$ s.t. $f(x)=c$.
\end{theorem}

\begin{property}
    Let $(X,d)$ be a metric space with the discrete metric.
    Let $E$ be a subset of $X$ with at least two elements. Then $E$ is disconnected.
\end{property}
So any connected subset with discrete metric must be a singleton. 
(Empty set is a little tricky, usually we define it as connected.)

\begin{corollary}
    Let $(X,d)$ be a metric space with the discrete metric.
    A function $f:X\rightarrow Y$ is continuous iff it is constant.
\end{corollary}

Some theorem is in progress.


\section{Topological Space}


\begin{exercise}
    Let $(X,\mathcal{F})$ be a topological space.
    \begin{enumerate}
        \item Write down the definition of a power set.
        \item Write down the definition of a topological space. 
    \end{enumerate}
\end{exercise}

\begin{definition}[Neighborhood]
    We say a neighborhood of $x$ is any open set $V\in\mathcal{F}$ s.t. $x\in V$.
\end{definition}

\begin{exercise}
    With the definition of neighborhood, write down the defintion of interior, exterior, boundary, and adherent point.
\end{exercise}


\begin{definition}[Continuity]
    Let $(X,\mathcal{F})$ and $(Y,\mathcal{G})$ be two topological spaces, and $f: X\rightarrow Y$ be a function.
    $f$ is continuous at $x_0\in X$ if for every nbhd $V\in \mathcal{G}$ of $f(x_0)$, $\exists$ a nbhd $U\in \mathcal{F}$ of $x_0$ s.t. $U\subseteq f^{-1}(V).$ 
\end{definition}

\begin{definition}[Topological Convergence]
    Let ${x_n}^\infty_1$ be a sequence in $X$. We say $x_n\rightarrow x$ if for every nbhd of $x$, called $V$, $\exists N\in\mathbb{N}$ s.t. $x_n\in V$ for $n\geq N$.
\end{definition}

\begin{exercise}
Let a sequence $\{x_n\}^\infty_{n=1}$ does not converge to $x$ for any $x\in X$.
That is, $\forall x\in X,\, \exists$, a nbhd of $x$, called $u_x$ s.t. $u_x$ only contains finite many points.
Negatate the statement you will get the definition of convergence. 
\end{exercise}

\begin{exercise}
    Write down the definition of a Hausdorff space.
\end{exercise}

\begin{theorem}
    Suppose $(X,\mathcal{F})$ is a Hausdorff space. Then every convergent sequence in $X$ converges to a unique limit.
\end{theorem}

\begin{pf}
    Suppose $\exists x\neq y$ s.t. $x_n\rightarrow x$ and $x_n\rightarrow y$. Use the defintion of Hausdorff space and find the contradiction (non-empty intersection).
\end{pf}



\section{Function Sequence and Convergence}

Let $\{f_n(x)\}^\infty_{n=1}$ be a sequence of functions from $(X,d_X)$ to $(Y,d_Y)$.
Let $f:X\rightarrow Y$ be an another function.
\begin{definition}[Pointwise Convergence]
We say $f_n\rightarrow f$ pointwise if for every $x\in X$ (fixed), $\exists\,N_x$ s.t. $d_Y(f_n(x), f(x))<\epsilon$ for $n\geq N_x$ and $\epsilon>0$.
\end{definition}

\begin{definition}[Uniform Convergence]
We say $f_n\rightarrow f$ uniformly if $\exists\, N$ s.t. $d_Y(f_n(x), f(x))<\epsilon$ for $n\geq N$ and $\epsilon>0$ for any $x\in X$ (not only a fixed point).
\end{definition}

\begin{theorem}
    Suppose $f_n$ is continous at $x_0$. If $f_n\rightarrow f$ uniformly, then $f$ is continous at $x_0$.
\end{theorem}

\begin{pf}
    \begin{enumerate}
        \item $f_n\rightarrow f$ uniformly, write down the defintion with $\frac{\epsilon}{3}$.
        \item Fix $N_1\geq N$, $f_{N_1}$ is continuous at $x_0$, write down the defintion with $\frac{\epsilon}{3}$.
        \item By triangle inequality, we have $d_Y(f(x), f(x_0))<\epsilon$ for $d_X(x,x_0)<\delta$.
        \item Given $x_0\in X$ is arbitrary, we can further prove $f$ is continuous in $X$.
    \end{enumerate}
\end{pf}


\begin{theorem}
    Let $(Y,d_Y)$ be a complete metric space, and let $E\subseteq X$.
    Suppose $f_n$ is a sequence from $E$ to $Y$ that converges to some function $f:E\rightarrow Y$ uniformly.
    Let $x_0$ be an adherent point of $E$.
    Suppose that for each $n$,$\lim_{x\rightarrow x_0, x\in E}f_n(x)$ exists,
    then $\lim_{x\rightarrow x_0, x\in E}f(x)$ also exists. Moreover, 
    \[
\lim_{n\to\infty}\,\lim_{\substack{x\to x_0\\x\in E}} f_n(x)
\;=\;
\lim_{\substack{x\to x_0\\x\in E}}\,\lim_{n\to\infty} f(x).
\]
\end{theorem}

\begin{theorem}
    If $f_n\rightarrow f$ uniformly and $\lim_{n\rightarrow}x_n=x\in X$, then $\lim_{n\rightarrow\infty}f_n(x_n)=f(x).$ 
\end{theorem}

\begin{definition}[Bounded Function]
    $f$ is bounded if $\exists\, y_0\in Y$ and $R>0$ s.t. $f(X)\subseteq B_Y(y_0,R),$ i.e. $d_Y(f(x), y_0)<R$ for any $x\in X$.
\end{definition}

\begin{theorem}
    If $f_n$ is bounded for each $n$ and $f_n\rightarrow f$ uniformly, then $f$ is bounded.
\end{theorem}


\begin{definition}[Supremum Norm]
    Let $B(X\rightarrow Y)$ be the set of all bounded functions from $(X,d_X)$ to $(Y,d_Y)$.
    Define 
    \[
    d_{\infty}(f,g) := \sup_{x\in X} d_Y(f(x), g(x))=\norm{f(x)-g(x)}_{\infty}\quad \forall f,g\in B(X\rightarrow Y).
    \]
\end{definition}

\begin{theorem}
    $f_n\rightarrow f$ in $(B(X\rightarrow Y),d_{\infty})$ iff $f_n\rightarrow f$ uniformly.
\end{theorem}

\begin{theorem}
Let $C(X\rightarrow Y)$ be the set of bounded and continuous function. That is, $C(X\rightarrow Y)$ is a subspace induced by $d_\infty$ in $B(X\rightarrow Y)$.
If $Y$ is complete, then $(C(X\rightarrow Y),d_\infty)$ is a complete metric space.  
\end{theorem}


\begin{definition}[Partial Sum]
    \[S_N(x)=\sum_{i=1}^{N}f_i(x).\]
\end{definition}

\begin{exercise}
    Write down the definition of pointwise convergence of infinite series.
    \[\sum_{i=1}^{\infty}f_i=f \Longleftrightarrow \lim_{N\rightarrow\infty}S_N(x)=f\]
\end{exercise}


\begin{theorem}[Weierstrass M-test]
   Let $\{f_n(x)\}^\infty_{n=1}$ be a sequence of bounded, real-valued, continuous function on $X$.
   Suppose $\sum_{n=1}^{\infty}\norm{f_n}_\infty$ converges. Then $\sum_{n=1}^{\infty}f_n$ converges uniformly to some bounded, real valued, and continous function on $X$.
\end{theorem}


\subsection*{Uniform convergence, Integral and Derivative}
Firstly, uniform convergence allows us to exchange infinite summation and integration.

\begin{theorem}
Let $f_n:I([a,b])\rightarrow\mathbb{R}$ be a sequence of Riemann integrable function.
Suppose $f_n\rightarrow f$ uniformly, where $f:[a,b]\rightarrow\mathbb{R}$.
Then $f$ is Riemann integrable and $\lim_{n\rightarrow\infty}\int_{I}f_n = \int_{I}f$, where $I=[a,b].$   
\end{theorem}

\begin{corollary}
Let $f_n:[a,b]\rightarrow\mathbb{R}$ be a sequence of Riemann integrable function.
Suppose $\sum_{n=1}^{\infty}f_n$ converges uniformly, where $f(x)=\sum_{n=1}^{\infty}f_n(x)$.
Then $f$ is Riemann integrable and $\lim_{n\rightarrow\infty}\int_{I}f_n = \int_{I}f$, where $I=[a,b].$   
\end{corollary}

However, we require reversed side argument with additional assumptions to gurantee the uniform convergence of derivative.

\begin{theorem}
    Let $f_n:[a,b]\rightarrow\mathbb{R}$ be a differentiable function whose derivative $f'_n:[a,b]\rightarrow\mathbb{R}$ is continuous.
    Suppose $f'_n\rightarrow g$ uniformly, where $g:[a,b]\rightarrow\mathbb{R}$ is continous (Think why?).
    Additionally, suppose $\exists\, x_0\in [a,b]$ s.t. $\lim_{n\rightarrow\infty}f_n(x_0)$ exists.
    Then:
    \begin{enumerate}
        \item $f_n\rightarrow f$ uniformly.
        \item $f'=g.$ Informally:
        \[\frac{d}{dx}(\lim_{n\rightarrow\infty}f_n(x))= \lim_{n\rightarrow\infty}f'_n(x).\]
    \end{enumerate}
\end{theorem}


\begin{corollary}
    Let $f_n:[a,b]\rightarrow\mathbb{R}$ be a differentiable function whose derivative $f'_n:[a,b]\rightarrow\mathbb{R}$ is continuous.
    Suppose the series $\sum_{n=1}^{\infty}\norm{f'_n}$ converges absolutely (By M-test, what series converges uniformly ?).
    Additionally, suppose $\sum_{n=1}^{\infty}f_n(x_0)$ converges for some $x_0\in [a,b]$.
    Then:
    \begin{enumerate}
        \item $\sum_{n=1}^{\infty}f_n$ converges uniformly.
        \item Exchangeable btw derivative and infinite summation.
        \[\frac{d}{dx}(\sum_{n=1}^{\infty}f_n(x))= \sum_{n=1}^{\infty}f'_n(x).\]
    \end{enumerate}
\end{corollary}



\section{Power Series}

\begin{definition}[Power Series]
    Let $a$ be a real number. We say a (formal) power series centered at $a$ is any series of the form
    \[\sum_{n=0}^{\infty}c_n(x-a)^n,\]
    where $c_n$ is a sequence of real numbers independent of $x$.
\end{definition}

\begin{definition}[Radius of Convergence]
For any power series $\sum_{n=0}^{\infty
}c_n(x-a)^n,$ we define the radius of convergence $R$ of this series as:
\[R \coloneqq \frac{1}{\limsup_{n\rightarrow\infty}\lvert c_n\rvert^\frac{1}{n}}.\]
By conventions, $\frac{1}{0}=+\infty$ and $\frac{1}{+\infty}=0$.    
\end{definition}
\begin{theorem}[Ratio Test]
    Let $\{a_n\}$ be a sequence of real numbers and $\sum_{n=0}^{\infty}a_n$ be a series. Suppose $L\coloneqq\lim_{n\rightarrow\infty}\lvert\frac{a_{n+1}}{a_n}\rvert$ exists in $[0,\infty]$.
    \begin{enumerate}
        \item If $L<1$, the series converges absolutely.
        \item If $L>1$, the series diverges.
        \item If $L=1$, the convergence of series is indetermined.
    \end{enumerate}
\end{theorem}


\begin{theorem}
Given a power series $f(x)= \sum_{n=0}^{\infty}c_n(x-a)^n$ and radius of convergence $R$.
\begin{enumerate}
    \item If $|x-a|<R$, then $f$ converges absolutely.
    \item If $|x-a|>R$, then $f$ diverges.
    \item For $0<r<R$, $f$ converges uniformly on $[a-r, a+r].$
    \item If $f$ is differentiable when $|x-a|<R$, then $\sum_{n=0}^{\infty}n c_n(x-a)^{n-1}\rightarrow f'$ uniformly on $[a-r, a+r].$
    \item For any $[y,z]\subseteq(a-R,a+R),$ $\int_{z}^{y}f(x)dx=\sum_{n=0}^{\infty}\frac{c_n(z-a)^{n+1}-c_n(y-z)^{n+1}}{n+1}.$
\end{enumerate}
\end{theorem}

\begin{pf}
    For (4),  $\sum_{n=1}^{N}f_n$ is uniformly convergent as $N\rightarrow\infty$ since $|x-a|<r$; $\sum_{n=1}^{N}f'_N$ converges uniformly by ratio test, comparison test and Weierstrass M-test.
\end{pf}

\begin{theorem}
Let $(b_n)_n$ be a sequence such that $b_n>0\,\forall n\in\mathbb{N}$.
If $\lim_{n\rightarrow\infty}\frac{b_{n+1}}{b_n}=\ell$, then $\lim_{n\rightarrow\infty}\sqrt[n]{b_n} = \ell.$
\end{theorem}

This theorem shows that root test is a stronger version than ratio test (Think).

\begin{definition}[Real analytic function]
    Let $E\subseteq\mathbb{R},$ and $f:E\rightarrow\mathbb{R}$ be a function.
    Fix an interior point $a\in E$, we say $f$ is real analytic at $a$ if there exists a power series $\sum_{n=0}^{\infty}c_n(x-a)^n$ on a neighborhood of $a$ s.t.
    \[f(x)=\sum_{n=0}^{\infty}c_n(x-a)^n\]
    for all $x\in (a-r, a+r),$ where $r$ is the radius of convergence.
\end{definition}

\begin{remark}
    $f$ is real analytic at $a$ $\rightarrow$ its Taylor series at $a$ converges to $f$. $a$ is arbitrary on $E$.
\end{remark}

\begin{theorem}[Real analytic function is smooth]
    If $f$ is real analytic at $a$, then $f$ is inifinitely differentiable at $a$.
    That is, for every integer $k\geq 0$, the function is k-times differentiable on $(a-r, a+r).$
    The k-th derivative is given by:
    \begin{equation*}
    \begin{split}
        &\sum_{n=0}^{\infty}c_{n+k}\frac{(n+k)!}{n!}(x-a)^{n},\\
        &c_k = \frac{f^{(k)}(a)}{k!}.
    \end{split}
    \end{equation*}
\end{theorem}

\begin{corollary}[Uniqueness of power series expansions]
    If $f$ is real analytic at $a$ and admits two power series expansions:
    \[f(x)=\sum_{n=0}^{\infty}c_n(x-a)^a\quad \text{and}\quad f(x)=\sum_{n=0}^{\infty}d_n(x-a)^a\]
    then $c_n = d_n\,\forall n\geq 0.$
\end{corollary}

\begin{example}[$C^\infty$ does not imply real analytic]
Consider a bump function (Cauchy, 1823):
\[f(x)=\begin{cases}
    e^{-1/x^2},\, x\neq 0\\
    0,\, x=0.
\end{cases}\]
\begin{enumerate}
    \item f is inifinitely differentiable and $f^{(n)}(x)=0\,\forall n\in\mathbb{N}.$
    \item f is not real analytic at $x=0$ since its Taylor series does not equal f on any nhbd of $0$.
\end{enumerate}
\end{example}

\begin{theorem}[Abel's Theorem]
    Let $f(x)=\sum_{n=0}^{\infty}c_n(x-a)^n$ be a power series with r.o.c. $0<R<\infty.$
    \begin{enumerate}
        \item If the series converges at $x=a+R$, then $f$ is continuous at $x=a+R$, i.e. $\lim_{x\rightarrow a+R^-}\sum_{n=0}^{\infty}c_n(x-a)^n = \sum_{n=0}^{\infty}c_n R^n.$
        \item If the series converges at $x=a-R$, then $f$ is continuous at $x=a-R$, i.e. $\lim_{x\rightarrow a-R^+}\sum_{n=0}^{\infty}c_n(x-a)^n = \sum_{n=0}^{\infty}c_n (-R)^n.$
    \end{enumerate}
\end{theorem}

Note that the theorem only states the boundary behavior of power series.
For general series, convergence at the boundary does not imply continuity.


\subsection*{Exponential and Logarithm Function}


\begin{definition}[Exponential Function]
    For every real number $x$, define $\exp(x)\coloneq \sum_{n=0}^{\infty}\frac{x^n}{n!}$.
\end{definition}

\begin{property}
    The function $\exp(x)$ satisfies the following properties:
    \begin{enumerate}
        \item For every real number $x$, $\exp(x)$ is absolutely convergent. Moreover, $\exp(x)$ is real analytic on $(-\infty, \infty)$.
        \item $\frac{d}{dx}\exp(x)=\exp(x).$
        \item For each $[a,b]\subseteq \mathbb{R},\, \int_{a}^{b}\exp(x)dx = \exp(b)-\exp(a)$.
        \item For $x,y\in\mathbb{R},\, \exp(x+y)=\exp(x)\exp(y).$
        \item $\exp(0)=1,\,\exp(x)>0\,\forall x\in\mathbb{R},\, \exp(-x)=\frac{1}{\exp(x)}$.
        \item $\exp(x)>\exp(y)$ if $x>y$.
    \end{enumerate}
\end{property}


\begin{pf}
    \begin{enumerate}
        \item Use ratio test.
        \item Real analytic function is smooth, differentiate the power series term by term.
        \item Given (2), use FTC.
        \item Use the Cauchy product of two series.
        \item For the third part, use (4).
        \item Apply $\exp(x)$ on $x-y>0$, and use (5). 
    \end{enumerate}
\end{pf}


\begin{definition}[Euler Number]
    Define $e\coloneq \exp(1)=\sum_{0}^{\infty}\frac{1}{n!}$.
\end{definition}

\begin{theorem}
    For every real number $x$, $\exp(x)=e^x$.
\end{theorem}

\begin{pf}
    \begin{enumerate}
        \item For $x\in\mathbb{Z}$, use the property (4) and (5).
        \item For $x\in\mathbb{Q},$ let $x=\frac{p}{q}$, consider $(\exp(\frac{p}{q}))^q.$
        \item For $x\in\mathbb{R}\setminus\mathbb{Q},$ use sequential continuity.
    \end{enumerate}
\end{pf}



\begin{definition}[Logarithm Function]
    Define $\ln:(0,\infty)\rightarrow\mathbb{R}$ as the inverse function of $\exp(x)$.
\end{definition}

\begin{remark}
    First justify $\exp(x)$ is a bijection, so the inverse function exists.
\end{remark}

\begin{property}
    The function $\exp(x)$ satisfies the following properties:
    \begin{enumerate}
        \item $\frac{d}{dx}\ln(x)=\frac{1}{x}.$
        \item For each $(a,b)\subseteq \mathbb{R},\, \int_{a}^{b}\frac{1}{x}dx = \ln(b)-\ln(a)$.
        \item For $x,y\in (0,\infty),$ we have $\ln(x+y)=\ln(x)\ln(y).$
        \item $\ln(1)=0,\, \ln(\frac{1}{x})=-\ln(x)\,\forall x\in (0,\infty)$.
        \item For $x,y\in (0,\infty),\, \ln(x^y)=y\ln(x).$
        \item For $x\in (-1,1),$ $\ln(1-x)=-\sum_{n=1}^{\infty}\frac{x^n}{n}$.
        \item For $x\in (0,2),$ $\ln(x)=\sum_{n=1}^{\infty}\frac{(-1)^{n+1}(x-1)^n}{n}.$
        \item For any $a>0,$ $\ln(x)$ is real analytic at $a$, where 
        \[\ln(x)=\ln(a)+\sum_{n=1}^{\infty}\frac{(-1)^{n-1}}{a^n n}(x-a)^n.\]
    \end{enumerate}
\end{property}

\begin{pf}
    \begin{enumerate}
        \item Let $y=\ln(x)$, then $x=\exp(y)$. Derive $\frac{dx}{dx}=\frac{d\exp(y)}{dx}=\exp(y)\frac{dy}{dx}.$
        \item Exercise (FTC).
        \item LHS: $\exp(\ln(xy))=xy$;\\
              RHS: $\exp(\ln(x)+\ln(y))=\exp(\ln(x))\exp(\ln(y))=xy$. Note that $\exp(x)$ is injective.
        \item Use (3).
        \item LHS: $\exp(y\ln(x))=e^{y\ln(x)}=(e^{\ln(x)})^y = x^y$.\\
              RHS: $\exp(\ln(x^y))=x^y$. Then use $\exp(x)$ is injective.
        \item Consider $\frac{d\ln(1-x)}{dx}=\frac{-1}{1-x}$.
        \item Change of variable $z=1-x$.
        \item Change of variable $y=x-a$.
    \end{enumerate}
\end{pf}


\subsection*{Trigonometric Function}

\begin{definition}[Cosine and Sine Function]
    Suppose $z\in\mathbb{C}$. Define:
    \[
    \begin{cases}
        \cos(z)=\frac{e^{iz}+e^{-iz}}{2}\\
        \sin(z)=\frac{e^{iz}-e^{-iz}}{2},
    \end{cases}
    \]
    where \(i=\sqrt{-1}\).
\label{tri_exp_def}
\end{definition}

\begin{theorem}[Euler's Formula]
    For $z\in\mathbb{C},$ we have:
    \[
    \begin{cases}
     e^{iz}=\cos(z)+i\sin(z)\\
     e^{-iz}=\cos(z)-i\sin(z).    
    \end{cases}
    \]
\end{theorem}

\begin{pf}
    Derive from the above definition.
\end{pf}

\begin{theorem}[Power Series of Cosine and Sine]
    \begin{equation*}
    \begin{split}
        \cos(z)&=\sum_{n=0}^{\infty}\frac{(-1)^n z^{2n}}{(2n)!},\\
        \\
        \sin(z)&=\sum_{n=0}^{\infty}\frac{(-1)^n z^{2n+1}}{(2n+1)!}
    \end{split}
    \end{equation*}
\label{tri_power_series}
\end{theorem}

\begin{pf}
    Extend power series of exponential function to complex plane.
    Derive the power series of $e^{iz}$, by Euler's formula, compare the real part and imaginary part.
\end{pf}

In particular, if $z\in\mathbb{R}$, we can obtain the similar results from above.

\begin{property}
Let $x,y\in\mathbb{R}$, we have:
\begin{enumerate}
    \item $\sin^2(x) + \cos^2(x) = 1;\, \sin(x)\in [-1,1], \cos(x)\in [-1,1]$.
    \item $\sin^\prime(x)=\cos(x);\, \cos^\prime (x) = -\sin(x)$.
    \item $\sin(-x)=-\sin(x);\, \cos(-x)=\cos(x)$.
    \item $\sin(x+y)=\sin(x)\cos(y)+\cos(x)\sin(y);\\
           \cos(x+y)=\cos(x)\cos(y)-\sin(x)\sin(y)$ 
    \item $\sin(0)=0, \, \cos(0)=1$
\end{enumerate}
\end{property}

\begin{pf}
    \begin{enumerate}
        \item For (1) and (2), compute directly using definition \eqref{tri_exp_def}.
        \item For (3), use \eqref{tri_power_series}. This shows that sine is an odd function and cosine is an even function.
        \item For (4), consider $e^{i(x+y)} = e^{ix}e^{iy}= \cos(x+y) + i\sin(x+y).$ 
        \item For (5), use \eqref{tri_power_series}.        
    \end{enumerate}
\end{pf}

\begin{exercise}
    Basic trigonometric identities
    \begin{enumerate}
        \item $\sin(\pi-x)=\sin(x), \, \cos(\pi-x)=-\cos(x)$.
        \item $\sin(x+\pi)=-\sin(x), \, \cos(\pi+x)=-\cos(x)$.
        \item $\sin(x+2\pi)=\sin(x), \, \cos(x+2\pi)=\cos(x)$ ($2\pi$ periodic).
        \item $\sin(x \pm \tfrac{\pi}{2}) = \pm \cos(x), \, \cos(x\pm\tfrac{\pi}{2}) = \mp\sin(x)$.
        \item $\sin(2x)=2\sin(x)\cos(x), \, \cos(2x)=\cos^2(x)-\sin^2(x)$.
    \end{enumerate}
\end{exercise}



\section{Fourier Series}

\begin{definition}[Periodic Function]
    Let $L$ be a positive real number.
    A function $f:\mathbb{R}\rightarrow\mathbb{C}$ is said to be $L-$ periodic, 
    if for every $x\in\mathbb{R},\, f(x+L)=f(x).$
\end{definition}


\begin{example}
    $e^{2\pi inx}$ is 1-periodic, and so are $\cos(2\pi nx), \sin(2\pi nx)$.
\end{example}


\begin{remark}
    $1-$periodic functions is also called $\mathbb{Z}-$periodic functions.
\end{remark}

\begin{definition}[Quotient Space $\mathbb{R}/\mathbb{Z}$]
    Define an equivalence relation:
    $x\sim y \Leftrightarrow x-y\in\mathbb{Z}$.
    Then the equivalence class $[x]$ is given by:
    $[x]=\{y:x-y\in\mathbb{Z}\}$.
    So the quotient space $\mathbb{R}/\mathbb{Z} \cong [0,1)$.
\end{definition}

\begin{definition}[Supremum Metric on $C(\mathbb{R}/\mathbb{Z}; \mathbb{C})$]
 The set of continous complex-valued, $1-$periodic functions on $\mathbb{R}/\mathbb{Z}$ is denoted by $C(\mathbb{R}/\mathbb{Z}; \mathbb{C})$.
 Take $f\in C(\mathbb{R}/\mathbb{Z}; \mathbb{C})$, $f(x)=f^1(x)+if^2(x),$ where $f^1, f^2$ are real-valued continous functions.
 Define the supremum norm:
 \begin{align*}
     d_\infty(f,g)&=\sup_{x\in\mathbb{R}}|f(x)-g(x)|\\
                  &=\sup_{[0,1]} |f(x)-g(x)|\\
                  &=\sup_{[0,1]}\sqrt{(f^1(x)-g^1(x))^2+(f^2(x)-g^2(x))^2}.
 \end{align*}

\end{definition}

\begin{theorem}
    If $f\in C(\mathbb{R}/\mathbb{Z}; \mathbb{C})$, then $f$ is bounded.
\end{theorem}

\begin{pf}
     Take $f\in C(\mathbb{R}/\mathbb{Z}; \mathbb{C})$, $f(x)=f^1(x)+if^2(x).$
     For each $i=1,2$, there exists $M_i>0$ s.t. $|f^i(x)|\le M_i$ by EVT.
     So $|f(x)|=\sqrt{(f^1(x))^2 + (f^2(x))^2}\le \sqrt{M^2_1 + M^2_2}$
\end{pf}


\begin{theorem}
    If $f\in C(\mathbb{R}/\mathbb{Z}; \mathbb{C})$, then $f$ is uniform continous.
\end{theorem}

\begin{theorem}
Basic properties of $C(\mathbb{R}/\mathbb{Z}; \mathbb{C})$:
    \begin{enumerate}
        \item $C(\mathbb{R}/\mathbb{Z}; \mathbb{C})$ over $\mathbb{C}$ is a vector space.
        \item $(C(\mathbb{R}/\mathbb{Z}; \mathbb{C}), d_\infty)$ is a complete metric space.
    \end{enumerate}
\end{theorem}

\begin{pf}
    \begin{enumerate}
        \item Verify the basic 10 axioms from linear algebra.
        \item The plan goes as following:
        \begin{enumerate}
            \item Take any Cauchy $(f_n)^{\infty}_{n=1}$ in $C(\mathbb{R}/\mathbb{Z}; \mathbb{C})$ and deduce the Cauchy sequence of $(f^i_n)^{\infty}_{n=1}$ in $C(\mathbb{R}/\mathbb{Z}; \mathbb{R}),\, i=1,2$.
            \item Use the completeness of $(C(\mathbb{R}/\mathbb{Z}; \mathbb{R}) ,d_\infty)$ to show $f^n\rightarrow f$ uniformly.
            \item Show $f$ is continuous, we need to fix a sufficient large $N$ and arbitrary $x_0\in [0,1]$.
            \item Show $f$ is $1-$periodic by $|f(x+1)-f(x)|\le |f_n(x+1)-f(x)| + |f_n(x+1)-f_n(x)| + |f_n(x)-f(x)|.$
        \end{enumerate}
    \end{enumerate}
\end{pf}

\begin{definition}[Inner Product]
Let $f,g\in C(\mathbb{R}/\mathbb{Z}; \mathbb{C})$, 
we define $\langle \cdot, \cdot \rangle: C(\mathbb{R}/\mathbb{Z}; \mathbb{C})\times C(\mathbb{R}/\mathbb{Z}; \mathbb{C}) \rightarrow \mathbb{C}$ by:
\[
\langle f, g \rangle = \int_0^1 f(x) \, \overline{g(x)} \, dx.
\]
\end{definition}

\begin{property}
Let $f,g,h\in C(\mathbb{R}/\mathbb{Z}; \mathbb{C})$:
\begin{enumerate}
    \item (Hermitian property) $\langle g,f \rangle = \langle \overline{f,g} \rangle$
    \item (Positivity) $\langle f,f \rangle \geq 0$. In particular, $\langle f,f \rangle = 0 \leftrightarrow f=0$.
    \item (Linearity in 1st component) $\langle f+g, h \rangle = \langle f,h \rangle + \langle g,f \rangle, \quad \langle cf,g \rangle = c\langle f,g \rangle$.
    \item (Anti-linearity in 2nd component) $\langle f, g+h \rangle = \langle f,g \rangle + \langle f,h \rangle, \quad \langle f,cg \rangle = \overline{c}\langle f,g \rangle$.
\end{enumerate}
\end{property}


\begin{definition}[$L^2$ Norm]
    For $f\in C(\mathbb{R}/\mathbb{Z}; \mathbb{C})$, we define
    \[
    \|f\|_{2} \coloneq \sqrt{\langle f,f\rangle}= \left(\int_0^1 f(x) \, \overline{f(x)} \, dx \right)^{1/2} = \left(\int_0^1 |f(x)|^2 \, dx\right)^{1/2}.
    \]
\end{definition}


\begin{theorem}
    Let $f,g\in C(\mathbb{R}/\mathbb{Z}; \mathbb{C})$:
    \begin{enumerate}
        \item (Non-degeneracy) $\|f\|_{2} = 0 \leftrightarrow \langle f,f \rangle = 0 \leftrightarrow f=0$.
        \item (Cauchy-Schwarz inequality) $|\langle f,g\rangle|\leq \|f\|_{2} \|g\|_{2}.$
        \item (Triangle inequality) $\|f+g\|_{2}\le \|f\|_{2} + \|g\|_{2}.$
        \item (Pythagora's theorem) If $\langle\ f,g \rangle = 0$, then $\|f+g\|_{2}^2 = \|f\|_{2}^2 + \|g\|_{2}^2$.
        \item For any $c\in\mathbb{C}, \|cf\|_{2} = |c|\|f\|_{2}.$
    \end{enumerate}
\end{theorem}

\begin{pf}
Property (1),(5) can be derived through direct compuation.
For Cachy-Schwarz inequality, we prove the real case first.
\begin{enumerate}
    \item For any $\lambda\in\mathbb{R}$, consider $g(\lambda) = \int_{0}^{1}(u(x)-\lambda v(x))^2 dx>0\,\forall u,v\in C(\mathbb{R}/\mathbb{Z}; \mathbb{R}).$ 
    \item Note $g>0$ and with positive leading coefficient, then we can derive the Cauchy inequality for real case.
    \item Next, show $|\int_{0}^{1}f(x)dx|\leq \int_{0}^{1}|f(x)|dx\,\forall f\in C(\mathbb{R}/\mathbb{Z}; \mathbb{C}).$
     Let $I = \int_{0}^{1}f(x)dx$ and $c=\frac{\overline{I}}{|I|}$, $cI=|I|\in\mathbb{R}$.
     Since $cI=\int_{0}^{1}\operatorname{Re}(cf(x))dx + \int_{0}^{1}\operatorname{Im}(cf(x))dx,$ which implies $Im(cf(x))=0.$
     $|cI| = |\operatorname{Re}(cf)|\leq|cf(x)| =(\operatorname{Re}(cf)^2 + \operatorname{Im}(cf)^2 )^{1/2}$.
     So $|I| = |cI| = |\int_{0}^{1}\operatorname{Re}(cf(x))dx|\le \int_{0}^{1}|\operatorname{Re}(cf(x))|dx\le \int_{0}^{1}|cf(x)|dx = \int_{0}^{1}|c||f(x)|dx,$ where $|c|=1$.
    \item Now consider $|\int_{0}^{1}f\bar{g}dx|\le \int_{0}^{1}|f||g|dx = \left|\int_{0}^{1}|f||g|dx\right|<(\int_{0}^{1}f^2 dx)^{1/2} + (\int_{0}^{1}g^2 dx)^{1/2}.$  
     This completes the proof of complex version.
\end{enumerate}
For triangle inequality, we prove property (3) and (4) is just a special case:
\begin{enumerate}
    \item Calculate $\|f+g\|_{2}^2$ and obtain $\langle f,g\rangle + \langle g, f\rangle$.
    \item Consider $\langle f,g\rangle + \langle g, f\rangle = 2|\operatorname(Re)(\langle f,g\rangle)|\le 2|\langle f,g\rangle|\le 2\|f\|_{2}\|g\|_{2}$.
    \item Note that $\|f+g\|_{2}^2=\|f\|_{2}^2+\|g\|_{2}^2+\langle f,g\rangle + \langle g, f\rangle \leq \|f\|_{2}^2+\|g\|_{2}^2 + 2\|f\|_{2}\|g\|_{2} = (\|f\|_{2}^2+\|g\|_{2}^2)^2$.
\end{enumerate}
\end{pf}



\begin{definition}[$L^2$ metric]
    Let $f,g\in C(\mathbb{R}/\mathbb{Z}; \mathbb{C})$
    We define
    \[d_{L^2}(f,g):
    \|f-g\|_{2} = (\int_{0}^{1}|f(x)-g(x)|^2 dx)^{1/2}
    \]
\end{definition}

\begin{exercise}
    Show that $(C(\mathbb{R}/\mathbb{Z}; \mathbb{C}), d_{L^2})$ is not complete by a counterexample.
\end{exercise}



\begin{definition}[Trigonometric Polynomial]
A function $f\in C(\mathbb{R}/\mathbb{Z}; \mathbb{C})$ is called trigonometric polynomial is the form of
\[f(x)=\sum_{n=-N}^{N}c_n e_n(x),\] where:
\begin{enumerate}
    \item $e_n(x)=e^{2\pi inx}$ is called the charcter of frequency $n$ and $1-$ periodic.
    \item $c_n=\hat{f}(n)=\int_{0}^{1}f(x)e^{-2\pi inx}dx$.
\end{enumerate}
\end{definition}

\begin{theorem}
    For any integers $n,m$, we have
    \[\langle e_n(x), e_m(x)\rangle=
    \begin{cases}
        1, \text{ if } i=j,\\
        0, \text{ if } i\neq j.
    \end{cases}
    \]
    This shows that charcters form an orthonormal basis for Fourier series.
\end{theorem}

\begin{definition}[Fourier Transformation]
Let $f\in  C(\mathbb{R}/\mathbb{Z}; \mathbb{C})$.
The $n$-th Fourier coefficient of $f$ is defined by:
\[
\hat{f}(n) = \int_{0}^{1}f(x)e^{-2\pi inx}dx.
\] 
The mapping $\hat{f}:\mathbb{Z}\rightarrow\mathbb{C}$ is called the Fourier transformation of $f$.
\end{definition}


\begin{theorem}
    Let $f(x)=\sum_{n=-N}^{N}c_n e_n(x)$ be a trigonometric polynomial, 
    where 
    \[c_n=\hat{f}(n) = \langle f,e_n\rangle = \int_{0}^{1}f(x)e^{-2\pi inx} dx.\]
    Then we have:
    \begin{enumerate}
        \item (Inversion Formula) $f(x)=\sum_{n=-N}^{N}\hat{f}(n)e_n = \sum_{n=-N}^{N}\hat{f}(n)e^{2\pi inx} = \sum_{n=-\infty}^{\infty}\hat{f}(n)e^{2\pi inx}$.
        \item (Plancherel Formula) $\|f\|^2_2 = \sum_{n=-N}^{N}|c_n|^2 = \sum_{n=-N}^{N}|\hat{f}(n)|^2 = \sum_{n=-\infty}^{\infty}|\hat{f}(n)|^2.$
    \end{enumerate}
\end{theorem}

\begin{remark}
    If $f$ is a trigonometric polynomial, it has only finitely many non-zero Fourier coefficients by construction. In particular,
    \[\hat f(k) =
    \begin{cases}
        c_k, & |k| \le N, \\
        0, & |k| > N.
\end{cases}\]
So the infinite sum is equivalent to finite sum.
\end{remark}

\begin{definition}[Fourier Series for 1-Periodic]
 The Fourier series of $f\in C(\mathbb{R}/\mathbb{Z}; \mathbb{C})$ is defined by:
 \[
  F(x)=\frac{1}{2}a_0 + \sum_{n=1}^{\infty} \left(a_n \cos(2\pi nx)+b_n \sin(2\pi nx)\right),
 \] 
 where:
 \begin{enumerate}
    \item $a_0=2\int_{0}^{1}f(x)dx,$
    \item $a_n=2\int_{0}^{1}f(x)\cos(2\pi nx)dx,$
    \item $b_n=2\int_{0}^{1}f(x)\sin(2\pi nx)dx$.
 \end{enumerate}
\end{definition}

\begin{remark}
 The definition of Trigonometric polynomial and truncated Fourier series are equivalent.
 We can fix a integer $N$, express $e^{2\pi inx}=\cos(2\pi nx)+i\sin(2\pi nx)$ and $e^{2\pi i(-n)x} = \cos{2\pi nx}-i\sin(2\pi nx)$.
 Additionally, we can use the change of variable to express Fourier series for $2\pi-$ periodic.
 Let $t=2\pi x$, $G(t)=F(\frac{t}{2\pi})$ and $g$ is $2-\pi$ periodic, we have:
 \[
  G(t)=\frac{1}{2}A_0 + \sum_{n=1}^{N} \left(A_n \cos(nt)+B_n \sin(nt)\right),
 \] 
 where
 \begin{enumerate}
    \item $A_0=\frac{2}{\pi}\int_{0}^{2\pi}g(t)dt,$
    \item $A_n=\frac{1}{\pi}\int_{0}^{2\pi}g(t)\cos(nt)dt = \frac{1}{\pi}\int_{-\pi}^{\pi}g(t)\cos(nt)dt,$
    \item $B_n=\frac{1}{\pi}\int_{0}^{2\pi}g(t)\sin(nt)dt = \frac{1}{\pi}\int_{-\pi}^{\pi}g(t)\sin(nt)dt$.
 \end{enumerate}

 Alternatively, we have:
 \[G(t) = \sum_{-N}^{N} \tilde{c}_n e^{int},\]
 
where 
\[\tilde{c}_n = \frac{1}{2\pi}\int_{-\pi}^{\pi}g(t) e^{-int}dt.\]

The question is whether $G$ converges to $g$ in what sense ($L_2$, pointwise or uniform convergence) as $N\rightarrow\infty$ ?
\end{remark}


\begin{definition}[Periodic Convolution]
Let $f,g\in C(\mathbb{R}/\mathbb{Z}; \mathbb{C})$. 
Define the periodic convolution $f*g:\mathbb{R}\rightarrow\mathbb{C}$ by
\[
(f * g)(x) = \int_{0}^{1} f(y)\, g(x-y)\, dy.
\]
\end{definition}

\begin{exercise}
    Show that periodic convolution is a linear transformation.
\end{exercise}


\begin{property}
Let $f,g,h\in C(\mathbb{R}/\mathbb{Z}; \mathbb{C})$:
\begin{enumerate}
    \item (Closure under convolution) $f*g\in C(\mathbb{R}/\mathbb{Z}; \mathbb{C}).$
    \item (Commutativity) $f*g = g*f$
    \item (Bilinearity) Fix $h, \,(f+g)*h = f*h + g*h$; fix $f, \, f*(g+h)=f*g + f*h$; $c(f*g)=(cf)*g=f*(cg).$
\end{enumerate}    
\end{property}

\begin{pf}
Hint: To show $f*g$ is continous, we need to use $f$ is bounded and $g$ is uniform continous.
\end{pf}


\begin{theorem}
    For any $f\in C(\mathbb{R}/\mathbb{Z}; \mathbb{C}),$
    $f*e_n=\hat{f}(n)e_n(x),$ where $\hat{f}(n)=\int_{0}^{1}f(y)e^{-2\pi ny}dy$.
\end{theorem}

\begin{exercise}
Define $T_f:  C(\mathbb{R}/\mathbb{Z}; \mathbb{C})\rightarrow  C(\mathbb{R}/\mathbb{Z}; \mathbb{C})$ by
$(T_f g)(x)=(f*g)(x)$ with the fixed $f\in C(\mathbb{R}/\mathbb{Z}; \mathbb{C})$.
Show that $T$ is a linear map and $e_n$ is an eigenvector of  w.r.t eigenvalue $\hat{f}(n)$.
\end{exercise}

\begin{lemma}
    Let $P = \sum_{-N}^{N}c_n e_n$ be a trigonometric polynomial, then $(f*P)(x) = \sum_{-N}^{N}c_n\hat{f}(n)e_n$.
\end{lemma}


\begin{definition}
    Let $\epsilon>0$ and $0<\delta<\frac{1}{2}$. A function $f\in C(\mathbb{R}/\mathbb{Z}; \mathbb{C})$ is called periodic $(\epsilon, \delta)$ approximation to identity if:
    \begin{enumerate}
        \item $f(x)>0\,\forall x\in\mathbb{R}$ and $\int_{0}^{1} f(x)=1$.
        \item $f(x)<\epsilon\, \forall x\in[\delta, 1-\delta]$.
    \end{enumerate}
\end{definition}

\begin{definition}[Fejer kernal]
    For each integer $N\geq 1$, define
    \[F_N(x)\coloneq \sum_{-N}^{N}\left(1-\frac{|n|}{N}\right)e_n.\]
    Alternatively, the Fejer can be written as
    \[F_N(x) = \frac{1}{N}\left|\sum_{k=0}^{N-1}e_k(x)\right|^2\]
    since
    \begin{equation*}
        \begin{split}
            F_N(x) &= \frac{1}{N}\sum_{k=0}^{N-1}e_k(x)\overline{\sum_{l=0}^{N-1}e_l(x)}\\
             &= \frac{1}{N}\sum_{k=0}^{N-1}\sum_{l=0}^{N-1}e_k(x)\overline{e_l(x)}\\
             &= \frac{1}{N}\sum_{k=0}^{N-1}\sum_{l=0}^{N-1}e_{k-l}(x)\\
             &= \frac{1}{N}\sum_{m=-(N-1)}^{N-1}\left(N-|m|\right)e_m(x)=\sum_{m=-(N-1)}^{N-1}\left(1-\frac{|m|}{N}\right)e_m(x), 
        \end{split}
    \end{equation*}
where $m=k-l$.
\end{definition}


\begin{remark}
    For $x\in\mathbb{Z}$,
    \[F_N(x) = \frac{1}{N}\left|\sum_{k=0}^{N-1}e_k(x)\right|^2 = F_N(x) = \frac{1}{N}\left|\sum_{k=0}^{N-1}1\right|^2 = N.\]
    
    For $x \notin\mathbb{Z}$,
    We can simplify $F_N(x)$ by computing:
    \begin{align*}
    \sum_{k=0}^{N-1}e_k(x)&=\sum_{k=0}^{N-1}e^{2\pi ikx} = \frac{e^{2\pi iNx}-1}{e^{2\pi ix}-1}\\
     &= \frac{e^{\pi iNx}-e^{-\pi iNx}}{e^{\pi ix}-e^{-\pi ix}}\frac{e^{\pi iNx}}{e^{\pi ix}} = e^{\pi i(N-1)x}\frac{\sin(\pi Nx)}{\sin(\pi x)}. 
    \end{align*}

    So 
    \begin{align*}
        F_N(x) &= \frac{1}{N}\left|\sum_{k=0}^{N-1}e_k(x)\right|^2\\
        & = \frac{1}{N}\underbrace{\left|e^{2\pi i(N-1)x}\right|}_{1}\left|\frac{\sin^2(\pi Nx)}{\sin^2(\pi x)}\right| 
    \end{align*}

\end{remark}


\begin{theorem}
    Let $\epsilon>0$ and $0<\delta<\frac{1}{2}$.
    There exists a trigonometric polynomial $P$ which is a periodic $(\epsilon, \delta)$ approximation to identity.
\end{theorem}

\begin{pf}
    Consider the Fejer kernal. Want to show Fejer kernal is indeed a approximation to identity.
\begin{enumerate}
    \item Compute 
    \[
    \int_{0}^{1}F_N(x)dx =  \sum_{-N}^{N}\left(1-\frac{|n|}{N}\right)e_n = (1-\frac{0}{N})\cdot 1 = 1,
    \]

    where
    \[
    \int_{0}^{1}e^{2\pi inx} dx = \begin{cases}
        1, \text{ if } n = 0,\\
        0, \text{ if } n \neq 0.
    \end{cases}
    \]

    \item Fix $\epsilon>0,\, 0<\delta<\frac{1}{2}$ and $\delta<|x|<\frac{1}{2}<1-\delta$.
    Derive $\sin(\pi\delta)<\sin(\pi x)<1,$ then we have:
    \[
    F_N(x) = \frac{1}{N}\left|\frac{\sin^2(\pi Nx)}{\sin^2(\pi x)}\right|\le \frac{1}{N}\frac{1}{\sin^2(\pi\delta)},
    \]
    we can choose sufficient large $N$ to satisfy the inequality.
\end{enumerate}
\end{pf}

\begin{theorem}[Weierstrass Approximation for Trigonometric Polynomial]
    Let $f\in C(\mathbb{R}/\mathbb{Z}; \mathbb{C})$. $\forall\epsilon>0,\,\exists$ a trigonometric polynomial $P\in C(\mathbb{R}/\mathbb{Z}; \mathbb{C})$
    s.t. $\| f-p \|_{\infty}<\epsilon$.
\end{theorem}

\begin{remark}
    This theorem implies that the set of trigonometric polynomial is dense in $C(\mathbb{R}/\mathbb{Z}; \mathbb{C})$ w.r.t $d_\infty$.
\end{remark}


\begin{pf}
Fix $\epsilon>0$.
\begin{enumerate}
    \item $f$ is bounded (Why?), there exists $M>0$ s.t. $|f(x)|\leq M$.
    \item $f$ is uniformly continous (Why?), there exists $\delta>0$ s.t. $|f(x)-f(y)|<\frac{\epsilon}{4}$ whenever $|x-y|<\delta$.
    \item Let $P$ be a trigonometric polynomial which is $(\frac{\epsilon}{4M},\delta)$ approximation to the identity.
    \item $(f*P)(x)$ is also a trigonometric polynomial (Why?) and $(f*P)(x) = (P*f)(x)$.
    \item Goal:
    \begin{align*}
    &\left|f(x)-(f*P)(x)\right| = \left|f(x)-(P*f)(x)\right| \\
    &= \left| f(x) \int_{0}^{1} P(y)dy - \int_{0}^{1}P(y)f(x-y)dy\right|\\
    &\leq \int_{0}^{\delta}\left|f(x)-f(x-y)\right|P(y)dy \text{ (Use uniform continuity)}\\
    &+ \int_{\delta}^{1-\delta}\left|f(x)-f(x-y)\right|P(y)dy \text{ (Use Approximation to identity and boundedness)}\\
    &+\int_{1-\delta}^{1}\left|f(x)-f(x-y)\right|P(y)dy \text{ (Use uniform continuity)}\\
    &\le \frac{\epsilon}{4} + 2M\frac{\epsilon}{4M} + \frac{\epsilon}{4} =\epsilon.
    \end{align*}
\end{enumerate}
\end{pf}

\begin{theorem}[Fourier Theorem]
    For any $f \in C(\mathbb{R}/\mathbb{Z}; \mathbb{C})$, the series $\sum_{n\in\mathbb{Z}}\hat{f}(n)e_n$ converges to $f$ in $L^2$ metric,
    i.e. for $\epsilon>0,$
    \[\lim_{N\rightarrow\infty}\left\|f-\sum_{n=-N}^{N}\hat{f}(n)e^{2\pi inx}\right\|_{L^2}<\epsilon.\]
\end{theorem}

\begin{pf}
    \begin{enumerate}
        \item By Weierstrass approximation, there exists a trig polynomial $P$ s.t. $\| f-p \|_{\infty}<\epsilon$, which implies $\| f-p \|_{L^2}<\epsilon$.
        \item Show that $\langle f-F_N, e_k\rangle = 0$ for each integer $k$, i.e. $f-F_N\perp \{e_k\}^{N}_{k=-N}$.
        \item For $N\ge N_0$, $P$ and $F_N$ lie in the subspace spanned by $\{e_k\}^{N}_{k=-N}$. Show that $\langle f-F_N, P-F_N\rangle = 0.$
        \item By Pythagoras theorem, $\left\|f-F_N\right\|^2_{L^2} = \left\|f-P\right\|^2_{L^2} + \left\|P-F_N\right\|^2_{L^2} \Rightarrow \left\|f-F_N\right\|_{L^2} \le \left\|f-P\right\|_{L^2} \le \left\|f-P\right\|_{\infty}<\epsilon.$
    \end{enumerate}
\end{pf}

\begin{theorem}
    If $f\in  C(\mathbb{R}/\mathbb{Z}; \mathbb{C})$ and the Fourier coefficients of $f$ s.t.
    $\sum_{n=-\infty}^{\infty}\left|\hat{f}(n)\right|<\infty$, then the Fourier series $F_N = \sum_{n=-N}^{N}\hat{f}(n)e_n$ converges to $f$ uniformly.
\end{theorem}

\begin{pf}
\begin{enumerate}
    \item $\sum_{n=-N}^{N}\left|\hat{f}(n)e_n\right| = \sum_{n=-N}^{N}\left|\hat{f}(n)\right|$. (Why?)
    \item By assumption $\sum_{n=-N}^{N}\left|\hat{f}(n)e_n\right|<\infty$. Then by Weierstrass M-test, $F_n\rightarrow F$ uniformly for some $F\in C(\mathbb{R}/\mathbb{Z}; \mathbb{C}),$ which implies $F_n\in F$ in $L^2$ metric.
    \item By Fourier's theorem $F_N\rightarrow f$ in $L^2$ metric.    
    \item By uniqueness of limit or triangle inequality, $F=f$.
\end{enumerate}    
\end{pf}

\end{document}